\chapter{数式}\label{chap:chapter4}
逆運動学式では,ロボット車の移動速度から,左右のモータの速度へと変換を行う.
それぞれのモータの速度を回転数に変換を行う.
この回転数をPWM 信号として,モータへ入力して速度制御を行う.

ロボットの移動速度(並進速度)を$V$ [m/s],角速度を$\omega$ [rad/s],左右輪の速度を$v_L$,$v_R$ [m/s] 車輪直径を$d$ [m]とする.
このロボットは,差動駆動型のロボットであるため,運動学式より,(\ref{eq:ik})式が成り立つ.
\begin{equation}
    \label{eq:ik}
    \begin{bmatrix}
        V \\
        \omega \\
        R \\
    \end{bmatrix}
    =
    \begin{bmatrix}
        \dfrac{v_L + v_R}{2} \\
        \dfrac{v_R - v_L}{2 d} \\
        \dfrac{d(v_L + v_R)}{v_R - v_L} \\
    \end{bmatrix}
\end{equation}
基準座標系での速度と角速度は
\begin{equation}
    \label{eq:ik_base}
    \begin{bmatrix}
        \dot{x} \\
        \dot{y} \\
        \dot{\theta} \\
    \end{bmatrix}
    =
    \begin{bmatrix}
        v \cos{\theta} \\
        v \sin{\theta} \\
        \omega \\
    \end{bmatrix}
\end{equation}
となる.これより,左右輪のモータの角速度はつぎとなる.
\begin{equation}
    \label{eq:wheel_speed}
    \begin{bmatrix}
        v_L \\
        v_R \\
    \end{bmatrix}
    =
    \begin{bmatrix}
        V - d \omega \\
        V + d \omega \\
    \end{bmatrix}
\end{equation}
モータ速度からモータ回転数$n$ [rpm]を求める式は,
\begin{equation}
    \label{eq:v_rpm}
    n = \dfrac{60}{2 \pi r} \times v \times D \times \dfrac{1}{d}
\end{equation}
となる.このとき,$D$ は減速比であり,$v$ [m/s]には,左右輪の速度$v_L$,$v_R$ を代入する.
